\chapter{Environment}

\section{Users and Uses}

JEAC will be used by students in Prof. Mills' research courses, and by researchers working on analog applications.  Thus the application will be used primarily within the Computer Science Deptartment at Indiana University.  The application is a single-user application; concurrent usage will be limited only by the number of available EACs.

Users will come from various backgrounds.  Nonetheless, it is expected that users will be experienced with digital computers and modern operating systems.  Users will primarily be upper-level computer science students and researchers.  The goal of the project is to create an easy-to-use abstraction atop the analog computers.  It is safe to assume that most users will have little previous experience working with the EAC.

\section{Hardware and Software Platforms}


The team is interested in developing a tool that aids in learning how to manipulate the extended analog computer.  Thus, the application should be as transparent as possible, and will be designed to be platform independent.  JAVA provides the necessary platform-neutral tools to create a portable, easily installed application.

\section{Interaction with Other Systems}

JEAC is primarily a pedagogical tool.  It is designed to help students and researchers understand the basic mechanics of the extended analog computer.  It does not have much utility outside of a teaching environment.  However, the design of the application suggest the use of a hardware abtraction layer (HAL) that, once completed, may be useful to other JAVA-based EAC applications.

\section{Impact on Operations}

The application will be used to help acclimate student and researchers with the analog computing paradigm.  By bringing together simple interaction and near realtime visualization, the team hopes that users -- especially those who come are used to digital computers -- will be able to better understand how the extended analog computer operates.